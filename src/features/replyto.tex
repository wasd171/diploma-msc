\section{Implementated the \texttt{REPLYTO} command}
\label{sec:replyto}

\textit{Components affected}: \textbf{COM Handler} (Section \ref{sec:com_handler})

\textit{Motivation}: in order to create a truly interconnected DAQ system it is not enough to be able to send SCPI commands to our nodes. We also need to be able to use their responses for a feedback loop of any kind. By implementing this feature on the COM handler level it  would be possible to return a response of any node.

\textit{Requirements}: following the contract of the \texttt{REQUEST} Feature \ref{sec:request} the COM handler needs to support a \texttt{:REPLYTO} command. We would like to achieve a behaviour as described below:

\begin{itemize}
	\item Only one \texttt{:REPLYTO} command can be processed at a time. When it is executed the COM handler should continue accepting commands from the distributor. These commands would be buffered in memory until the pending \texttt{:REPLYTO} command completes.
	\item{
		Given the example from the Feature \ref{sec:request}
		\begin{verbatim}
			HV:REPLYTO("SEQUENCER:RESULT 17, %2"):OUTPUT:VOLTAGE?
		\end{verbatim}
		we will send \texttt{OUTPUT:VOLTAGE?} to the node and pause the further execution of the SCPI commands.
	}
	\item{
		Upon executing the \texttt{:REPLYTO} command the COM handler sets a time window based on the integer \texttt{scpiResponseTimeoutMs} value from the config file \texttt{conf\_n2comhandler.cfg}. We have 3 cases to handle:
		\begin{itemize}
			\item If nothing came over the SCPI connection with the node during the waiting interval then it means that \texttt{:REPLYTO} has timed out. We resume the execution of the SCPI commands from the distributor from the next command, if any.
			\item If a complete message that starts with \highlight{:} was received then we assume that the node is trying to communicate with other components of the system. Such message is forwarded to the distributor directly as is and does not influence the \texttt{:REPLYTO} feature flow.
			\item Otherwise if a complete message was received while we are still within the time window of waiting for response then we consider it to be an answer to the \texttt{:REPLYTO} being in flight.
			\item If a node answer that does not start with \highlight{:} was received outside of the \texttt{:REPLYTO} timeout the COM handler would skip it and print a warning.
		\end{itemize}
		\item{
			If a complete SCPI response was received in time then our next task is to parse it. A node might decide to return multiple comma-separated values. We split the message into chunks by using the rules below:
			\begin{itemize}
				\item Message is scanned from left to right
				\item Splittable parts are separated by commas \highlight{,}
				\item Escaped commas \highlight{\textbackslash ,} are not separators
				\item Commas that appear inside of a string are not separators
				\item String starts with a quotation mark \highlight{"}
				\item String might end with a quotation mark \highlight{"}. If there is no closing quotation mark, we consider the string to end at the end of the message. So in \texttt{"1,2,3} none of the commas would be a separator
				\item Escaped quotation marks \highlight{\textbackslash "} are not considered quotation marks
			\end{itemize}
		}
		\item{
			After splitting the message we need to extract the correct \texttt{answerValue} by using the format argument, in our example that would be \texttt{\%2}
			\begin{itemize}
				\item \texttt{\%0} means that no splitting is required and \texttt{answerValue} is set to the received message as a whole
				\item \texttt{\%n} means that \texttt{answerValue} is set to the \texttt{n}-th part of the split message with count starting with 1. If the message contained no separators then \texttt{\%0} and \texttt{\%1} produce the equal output. If the split message contains less than \texttt{n} parts we set \texttt{answerValue} to an empty unquoted string.
			\end{itemize}
		}
		\item After extracting the \texttt{answerValue} we are ready to send back the answer. Let's assume that \texttt{answerValue = 289}. Following our example, the COM handler will need to send \texttt{:SEQUENCER:RESULT 17, 289} over FIFO to the distributor.
	}
\end{itemize}

\textit{Implementation details}:

\begin{itemize}
	\item{
		In the \textit{requirements} above one can often see a combination of words \textbf{complete message}. Neither TCP/IP nor FIFO files guarantee that the message sent as one would be received as one. All SCPI messages, such as
		\begin{itemize}
			\item Commands from the distributor
			\item Answers to the distributor
			\item Command to the node
			\item Answers from the node
		\end{itemize}
		must use a newline \highlight{\textbackslash n} symbol to indicate the ending of the command. Internally we have 2 string buffers: for the commands from the distributor and the answers from the node. Every received message is placed into a corresponding buffer. On every tick we attempt to select one \textbf{complete} message by splitting once on the newline symbol \highlight{\textbackslash n} starting from the beginning of the buffer. In case of success this \textbf{complete} message is placed in a corresponding sequence and treated as a SCPI command/answer. The SCPI answers buffer is not cleaned on the \texttt{:REQUEST} timeout.
	}
	\item We consider a SCPI response to be an answer to \texttt{:REPLYTO} if it was received during the timeout window. Even though the default value of \texttt{scpiResponseTimeoutMs} being 5000 should be enough for the node to return the answer in time, we cannot guarantee it. Thus a very late answer that comes during the execution of the \textbf{next} \texttt{:REPLYTO} command would be sent as an answer to the distributor.
	\item{
		Combining both points we might get a very confusing situation:
		\begin{enumerate}
			\item COM handler executes first \texttt{:REPLYTO}
			\item Node sends back \texttt{verylonganswer\textbackslash n}
			\item The answer does not fit in a single package and the COM handler received only \texttt{very} and adds it to the answer buffer.
			\item Due to the network error the connection between the node and the COM handler is broken. First \texttt{:REPLYTO} times out.
			\item Network connection is reestablished
			\item COM handler executes second \texttt{:REPLYTO}
			\item Node sends back \texttt{shortanswer\textbackslash n}
			\item Depending on the node logic, 2 situations would be possible:
			\begin{itemize}
				\item Node tries to send the remaining \texttt{longanswer\textbackslash n} first. COM handler would have an answer buffer \texttt{verylonganswer\textbackslash n} and send it as a result of the second \texttt{:REPLYTO}. The received message \texttt{shortanswer\textbackslash n} would be appended to the buffer and depending on timing either ignored or sent as a reply to the third \texttt{:REPLYTO}
				\item Node skips the \texttt{longanswer\textbackslash n} and decides to respond only with \texttt{shortanswer\textbackslash n}. COM handler would have an answer buffer \texttt{veryshortanswer\textbackslash n} and return it as a result of the second \texttt{:REPLYTO}.
			\end{itemize}
		\end{enumerate}
		One can see that with a lot of requests and slow nodes that might become an issue, thus the COM handler would print a warning if its queue of the pending SCPI commands has more than one item.
	}
\end{itemize}